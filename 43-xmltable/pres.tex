\begin{frame}
	\frametitle{¿Qué es \texttt{XMLTABLE}?}
\vspace{-0.6cm}
\begin{itemize}
\item Funcionalidad definida por el estándar SQL
\begin{itemize} \item ISO 9075-14 (SQL/XML) \end{itemize}
\item ``Función de tabla''
\item Convierte datos XML a forma de ``tabla relacional''
\item<2-> Implementado en PostgreSQL 10 por Pavel Stěhule
\end{itemize}
\pause \raggedleft
\includegraphics[scale=0.25]{pavel.png}

\pause

\footnotesize
\raggedleft con fuerte ayuda de Álvaro Herrera
{\color{blue}\url{https://blog.2ndquadrant.com/xmltable-intro/}}

\end{frame}

\begin{frame}[fragile]
\frametitle{Sinopsis de sintaxis}
\vspace{-0.3cm}
\begin{lstlisting}
xmltable( [ XMLNAMESPACES(url_namespace
                              AS nombre_namespace [, ...]), ]
          expresion_de_fila
              PASSING [BY REF] expresion_documento [BY REF]
          COLUMNS nombre { tipo
                           [ PATH expresion_de_columna ]
                           [ DEFAULT expresion_default ]
                           [ NOT NULL | NULL ]
                         | FOR ORDINALITY }
                   [, ...]
          )
\end{lstlisting}
\footnotesize\color{blue}\url{https://www.postgresql.org/docs/devel/static/functions-xml.html}
\end{frame}

\begin{frame}[fragile]
\frametitle{XML de ejemplo}
\footnotesize
\begin{lstlisting}
CREATE TABLE hoteldata (hotels XML);
\end{lstlisting}
\lstset{language=XML}
\begin{lstlisting}
<hotels>
 <hotel id="mancha">
  <name>La Mancha</name>
  <rooms>
   <room id="201"><capacity>3</capacity>
      <comment>Vista del canal</comment>
   </room>
   <room id="202"><capacity>5</capacity></room>
  </rooms>
  <personnel>
   <person id="1025">
    <name>Ferdinando Quijana</name><salary currency="PTA">45000</salary>
   </person>
  </personnel>
 </hotel>
\end{lstlisting}
\end{frame}

\begin{frame}[fragile]
\frametitle{XML de ejemplo (2)}
\footnotesize
\lstset{language=XML}
\begin{lstlisting}
<hotel id="valpo">
 <name>Valparaiso</name>
 <rooms>
  <room id="201"><capacity>2</capacity>
      <comment>Muy ruidosa</comment></room>
  <room id="202"><capacity>2</capacity></room>
 </rooms>
 <personnel>
  <person id="1026"><name>Katharina Wuntz</name>
    <salary currency="EUR">50000</salary> </person>
  <person id="1027"><name>Diego Velazquez</name>
     <salary currency="CLP">1200000</salary> </person>
 </personnel>
</hotel>
</hotels>
\end{lstlisting}

\end{frame}

\begin{frame}[fragile]
\frametitle{Tabulizando el \texttt{XML}}
\lstset{language=SQL}
\begin{lstlisting}
SELECT xmltable.*
  FROM hoteldata,
       XMLTABLE ('/hotels/hotel/rooms/room' PASSING hotels
                 COLUMNS
          id FOR ORDINALITY,
          hotel_name text PATH '../../name' NOT NULL,
          room_id int PATH '@id' NOT NULL,
          capacity int,
          comment text PATH 'comment'
                       DEFAULT 'A regular room'
                );
\end{lstlisting}
\end{frame}

\begin{frame}[fragile]

	\frametitle{El resultado}
\footnotesize
\begin{lstlisting}
COLUMNS id FOR ORDINALITY,
        hotel_name text PATH '../../name' NOT NULL,
        room_id int PATH '@id' NOT NULL,
        capacity int,
        comment text PATH 'comment' DEFAULT 'Habitacion normal'
\end{lstlisting}

\lstset{language=XML}
\begin{lstlisting}
<room id="201"><capacity>3</capacity> <comment>Vista del canal</comment> </room>
\end{lstlisting}

\normalsize
\vspace{1cm}
\begin{tabular}{r | l | r | r | l}
\textit{id} & \textit{hotel\_name} & \textit{room\_id} & \textit{capacity} & \textit{comment} \\
\hline
1 & La Mancha & 201 & 3 & Vista del canal \\
2 & La Mancha & 202 & 5 & Habitacion normal \\
3 & Valparaíso & 201 & 2 & Muy ruidosa \\
4 & Valparaíso & 202 & 2 & Habitacion normal \\
\end{tabular}

\end{frame}

\begin{frame}[fragile]
	\frametitle{Namespaces en \texttt{XML}}
	\footnotesize
	\lstset{language=XML}
	\begin{lstlisting}
<h:html xmlns:xdc="http://www.xml.com/books"
         xmlns:h="http://www.w3.org/HTML/1998/html4">

 <h:head><h:title>Book Review</h:title></h:head>
 <h:body>
  <xdc:bookreview>
   <xdc:title h:style="font-family: sans-serif;">
     XML: A Primer</xdc:title>
   <h:table>
    <h:tr align="center">
     <h:td>Author</h:td><h:td>Price</h:td>
     <h:td>Pages</h:td><h:td>Date</h:td></h:tr>
    <h:tr align="left">
     <h:td><xdc:author>Simon St. Laurent</xdc:author></h:td>
     <h:td><xdc:price>31.98</xdc:price></h:td>
     <h:td><xdc:pages>352</xdc:pages></h:td>
     <h:td><xdc:date>1998/01</xdc:date></h:td>
    </h:tr>
   </h:table>
  </xdc:bookreview>
 </h:body>
</h:html>
\end{lstlisting}
\end{frame}

\begin{frame}[fragile]
\frametitle{\texttt{XMLNAMESPACES}}
\begin{lstlisting}
SELECT xmltable.* FROM bookrev,
    XMLTABLE(XMLNAMESPACES (
                  'http://www.xml.com/books' AS xdc,
                  'http://www.w3.org/HTML/1998/html4' AS h),
       'h:body/xdc:bookreview' PASSING doc
    COLUMNS title text PATH 'xdc:title',
	    author text PATH 'h:table/h:tr/h:td/xdc:author' );
\end{lstlisting}
\pause
\begin{tabular}{l | l}
\textit{title} & \textit{author} \\
\hline
\\     XML: A Primer & Simon St. Laurent \\
\end{tabular}

\end{frame}

\begin{frame}[fragile]
	\frametitle{\texttt{XMLTABLE} y el resto del \texttt{FROM}}
\lstset{language=SQL}
\begin{lstlisting}
  SELECT hotel, currency, sum(salary)
    FROM hoteldata,
XMLTABLE ('/hotels/hotel/personnel/person' PASSING hotels
       COLUMNS hotel text     PATH '../../name' NOT NULL,
                salary integer PATH 'salary' NOT NULL,
                currency text  PATH 'salary/@currency'
                             NOT NULL
) GROUP BY hotel, currency;
\end{lstlisting}
\vspace{1cm}
\begin{tabular}{l | l | r}
\textit{hotel} & \textit{currency} & \textit{sum} \\
\hline
Valparaíso & CLP & 1200000 \\
Valparaíso & EUR & 50000 \\
La Mancha & PTA & 45000 \\
\end{tabular}
\end{frame}

\begin{frame}
	\frametitle{Desarrollo futuro}

		\raggedleft \includegraphics<2>[scale=0.5]{korotkov.jpg}
	\begin{center}
		{\only <1>{El futuro: ¿qué viene después?}}
	\pause
	\Huge
		JSON\_TABLE
	\end{center}
\end{frame}

\begin{frame}
	\frametitle{Preguntas}

	\Huge
	\begin{center}
		¿?
	\end{center}
\end{frame}
